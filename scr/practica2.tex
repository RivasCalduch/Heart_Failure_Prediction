% Options for packages loaded elsewhere
\PassOptionsToPackage{unicode}{hyperref}
\PassOptionsToPackage{hyphens}{url}
%
\documentclass[
]{article}
\usepackage{lmodern}
\usepackage{amssymb,amsmath}
\usepackage{ifxetex,ifluatex}
\ifnum 0\ifxetex 1\fi\ifluatex 1\fi=0 % if pdftex
  \usepackage[T1]{fontenc}
  \usepackage[utf8]{inputenc}
  \usepackage{textcomp} % provide euro and other symbols
\else % if luatex or xetex
  \usepackage{unicode-math}
  \defaultfontfeatures{Scale=MatchLowercase}
  \defaultfontfeatures[\rmfamily]{Ligatures=TeX,Scale=1}
\fi
% Use upquote if available, for straight quotes in verbatim environments
\IfFileExists{upquote.sty}{\usepackage{upquote}}{}
\IfFileExists{microtype.sty}{% use microtype if available
  \usepackage[]{microtype}
  \UseMicrotypeSet[protrusion]{basicmath} % disable protrusion for tt fonts
}{}
\makeatletter
\@ifundefined{KOMAClassName}{% if non-KOMA class
  \IfFileExists{parskip.sty}{%
    \usepackage{parskip}
  }{% else
    \setlength{\parindent}{0pt}
    \setlength{\parskip}{6pt plus 2pt minus 1pt}}
}{% if KOMA class
  \KOMAoptions{parskip=half}}
\makeatother
\usepackage{xcolor}
\IfFileExists{xurl.sty}{\usepackage{xurl}}{} % add URL line breaks if available
\IfFileExists{bookmark.sty}{\usepackage{bookmark}}{\usepackage{hyperref}}
\hypersetup{
  pdftitle={Practica 2: Data Cleaning},
  pdfauthor={Jose Luis Rivas Calduch y Mariano Jiménez Barca},
  hidelinks,
  pdfcreator={LaTeX via pandoc}}
\urlstyle{same} % disable monospaced font for URLs
\usepackage[margin=1in]{geometry}
\usepackage{color}
\usepackage{fancyvrb}
\newcommand{\VerbBar}{|}
\newcommand{\VERB}{\Verb[commandchars=\\\{\}]}
\DefineVerbatimEnvironment{Highlighting}{Verbatim}{commandchars=\\\{\}}
% Add ',fontsize=\small' for more characters per line
\usepackage{framed}
\definecolor{shadecolor}{RGB}{248,248,248}
\newenvironment{Shaded}{\begin{snugshade}}{\end{snugshade}}
\newcommand{\AlertTok}[1]{\textcolor[rgb]{0.94,0.16,0.16}{#1}}
\newcommand{\AnnotationTok}[1]{\textcolor[rgb]{0.56,0.35,0.01}{\textbf{\textit{#1}}}}
\newcommand{\AttributeTok}[1]{\textcolor[rgb]{0.77,0.63,0.00}{#1}}
\newcommand{\BaseNTok}[1]{\textcolor[rgb]{0.00,0.00,0.81}{#1}}
\newcommand{\BuiltInTok}[1]{#1}
\newcommand{\CharTok}[1]{\textcolor[rgb]{0.31,0.60,0.02}{#1}}
\newcommand{\CommentTok}[1]{\textcolor[rgb]{0.56,0.35,0.01}{\textit{#1}}}
\newcommand{\CommentVarTok}[1]{\textcolor[rgb]{0.56,0.35,0.01}{\textbf{\textit{#1}}}}
\newcommand{\ConstantTok}[1]{\textcolor[rgb]{0.00,0.00,0.00}{#1}}
\newcommand{\ControlFlowTok}[1]{\textcolor[rgb]{0.13,0.29,0.53}{\textbf{#1}}}
\newcommand{\DataTypeTok}[1]{\textcolor[rgb]{0.13,0.29,0.53}{#1}}
\newcommand{\DecValTok}[1]{\textcolor[rgb]{0.00,0.00,0.81}{#1}}
\newcommand{\DocumentationTok}[1]{\textcolor[rgb]{0.56,0.35,0.01}{\textbf{\textit{#1}}}}
\newcommand{\ErrorTok}[1]{\textcolor[rgb]{0.64,0.00,0.00}{\textbf{#1}}}
\newcommand{\ExtensionTok}[1]{#1}
\newcommand{\FloatTok}[1]{\textcolor[rgb]{0.00,0.00,0.81}{#1}}
\newcommand{\FunctionTok}[1]{\textcolor[rgb]{0.00,0.00,0.00}{#1}}
\newcommand{\ImportTok}[1]{#1}
\newcommand{\InformationTok}[1]{\textcolor[rgb]{0.56,0.35,0.01}{\textbf{\textit{#1}}}}
\newcommand{\KeywordTok}[1]{\textcolor[rgb]{0.13,0.29,0.53}{\textbf{#1}}}
\newcommand{\NormalTok}[1]{#1}
\newcommand{\OperatorTok}[1]{\textcolor[rgb]{0.81,0.36,0.00}{\textbf{#1}}}
\newcommand{\OtherTok}[1]{\textcolor[rgb]{0.56,0.35,0.01}{#1}}
\newcommand{\PreprocessorTok}[1]{\textcolor[rgb]{0.56,0.35,0.01}{\textit{#1}}}
\newcommand{\RegionMarkerTok}[1]{#1}
\newcommand{\SpecialCharTok}[1]{\textcolor[rgb]{0.00,0.00,0.00}{#1}}
\newcommand{\SpecialStringTok}[1]{\textcolor[rgb]{0.31,0.60,0.02}{#1}}
\newcommand{\StringTok}[1]{\textcolor[rgb]{0.31,0.60,0.02}{#1}}
\newcommand{\VariableTok}[1]{\textcolor[rgb]{0.00,0.00,0.00}{#1}}
\newcommand{\VerbatimStringTok}[1]{\textcolor[rgb]{0.31,0.60,0.02}{#1}}
\newcommand{\WarningTok}[1]{\textcolor[rgb]{0.56,0.35,0.01}{\textbf{\textit{#1}}}}
\usepackage{graphicx,grffile}
\makeatletter
\def\maxwidth{\ifdim\Gin@nat@width>\linewidth\linewidth\else\Gin@nat@width\fi}
\def\maxheight{\ifdim\Gin@nat@height>\textheight\textheight\else\Gin@nat@height\fi}
\makeatother
% Scale images if necessary, so that they will not overflow the page
% margins by default, and it is still possible to overwrite the defaults
% using explicit options in \includegraphics[width, height, ...]{}
\setkeys{Gin}{width=\maxwidth,height=\maxheight,keepaspectratio}
% Set default figure placement to htbp
\makeatletter
\def\fps@figure{htbp}
\makeatother
\setlength{\emergencystretch}{3em} % prevent overfull lines
\providecommand{\tightlist}{%
  \setlength{\itemsep}{0pt}\setlength{\parskip}{0pt}}
\setcounter{secnumdepth}{-\maxdimen} % remove section numbering

\title{Practica 2: Data Cleaning}
\author{Jose Luis Rivas Calduch y Mariano Jiménez Barca}
\date{11 de diciembre de 2020}

\begin{document}
\maketitle

{
\setcounter{tocdepth}{3}
\tableofcontents
}
\hypertarget{descripciuxf3n-del-dataset.}{%
\subsection{1. Descripción del
dataset.}\label{descripciuxf3n-del-dataset.}}

El data set objeto de estudio esta tomado de la plataforma
\emph{Kaggle}. Esta es una comunidad en línea de científicos de datos y
profesionales del aprendizaje automático, actualmente es una subsidiaria
de \emph{Google LLC}.

El nombre del data set es \emph{Heart Failure Prediction} (ECV) que son
la principal causa de muerte a nivel mundial, cobrando un estimado de
17,9 millones de vidas cada año, lo que representa el 31\% de todas las
muertes en todo el mundo. La insuficiencia cardíaca es un evento común
causado por las enfermedades cardiovasculares y este conjunto de datos
contiene 12 características que se pueden usar para predecir la
mortalidad por insuficiencia cardíaca.

La mayoría de las enfermedades cardiovasculares se pueden prevenir
abordando los factores de riesgo conductuales como el consumo de tabaco,
la dieta poco saludable y la obesidad, la inactividad física y el
consumo nocivo de alcohol utilizando estrategias para toda la población.

Las personas con enfermedad cardiovascular o que se encuentran en alto
riesgo cardiovascular (debido a la presencia de uno o más factores de
riesgo como hipertensión, diabetes, hiperlipidemia o enfermedad ya
establecida) necesitan una detección y manejo precoces donde un modelo
de aprendizaje automático puede ser de gran ayuda.

\textbf{Tipos de variables}

\begin{Shaded}
\begin{Highlighting}[]
\KeywordTok{sapply}\NormalTok{(rawData, class)}
\end{Highlighting}
\end{Shaded}

\begin{verbatim}
##                      age                  anaemia creatinine_phosphokinase 
##                "numeric"                "integer"                "integer" 
##                 diabetes        ejection_fraction      high_blood_pressure 
##                "integer"                "integer"                "integer" 
##                platelets         serum_creatinine             serum_sodium 
##                "numeric"                "numeric"                "integer" 
##                      sex                  smoking                     time 
##                "integer"                "integer"                "integer" 
##              DEATH_EVENT 
##                "integer"
\end{verbatim}

\textbf{Descripción de las variables}

\begin{Shaded}
\begin{Highlighting}[]
\KeywordTok{str}\NormalTok{(rawData)}
\end{Highlighting}
\end{Shaded}

\begin{verbatim}
## 'data.frame':    299 obs. of  13 variables:
##  $ age                     : num  75 55 65 50 65 90 75 60 65 80 ...
##  $ anaemia                 : int  0 0 0 1 1 1 1 1 0 1 ...
##  $ creatinine_phosphokinase: int  582 7861 146 111 160 47 246 315 157 123 ...
##  $ diabetes                : int  0 0 0 0 1 0 0 1 0 0 ...
##  $ ejection_fraction       : int  20 38 20 20 20 40 15 60 65 35 ...
##  $ high_blood_pressure     : int  1 0 0 0 0 1 0 0 0 1 ...
##  $ platelets               : num  265000 263358 162000 210000 327000 ...
##  $ serum_creatinine        : num  1.9 1.1 1.3 1.9 2.7 2.1 1.2 1.1 1.5 9.4 ...
##  $ serum_sodium            : int  130 136 129 137 116 132 137 131 138 133 ...
##  $ sex                     : int  1 1 1 1 0 1 1 1 0 1 ...
##  $ smoking                 : int  0 0 1 0 0 1 0 1 0 1 ...
##  $ time                    : int  4 6 7 7 8 8 10 10 10 10 ...
##  $ DEATH_EVENT             : int  1 1 1 1 1 1 1 1 1 1 ...
\end{verbatim}

\textbf{Resumen descriptivo de las variables}

\begin{Shaded}
\begin{Highlighting}[]
\KeywordTok{summary}\NormalTok{(rawData)}
\end{Highlighting}
\end{Shaded}

\begin{verbatim}
##       age           anaemia       creatinine_phosphokinase    diabetes     
##  Min.   :40.00   Min.   :0.0000   Min.   :  23.0           Min.   :0.0000  
##  1st Qu.:51.00   1st Qu.:0.0000   1st Qu.: 116.5           1st Qu.:0.0000  
##  Median :60.00   Median :0.0000   Median : 250.0           Median :0.0000  
##  Mean   :60.83   Mean   :0.4314   Mean   : 581.8           Mean   :0.4181  
##  3rd Qu.:70.00   3rd Qu.:1.0000   3rd Qu.: 582.0           3rd Qu.:1.0000  
##  Max.   :95.00   Max.   :1.0000   Max.   :7861.0           Max.   :1.0000  
##  ejection_fraction high_blood_pressure   platelets      serum_creatinine
##  Min.   :14.00     Min.   :0.0000      Min.   : 25100   Min.   :0.500   
##  1st Qu.:30.00     1st Qu.:0.0000      1st Qu.:212500   1st Qu.:0.900   
##  Median :38.00     Median :0.0000      Median :262000   Median :1.100   
##  Mean   :38.08     Mean   :0.3512      Mean   :263358   Mean   :1.394   
##  3rd Qu.:45.00     3rd Qu.:1.0000      3rd Qu.:303500   3rd Qu.:1.400   
##  Max.   :80.00     Max.   :1.0000      Max.   :850000   Max.   :9.400   
##   serum_sodium        sex            smoking            time      
##  Min.   :113.0   Min.   :0.0000   Min.   :0.0000   Min.   :  4.0  
##  1st Qu.:134.0   1st Qu.:0.0000   1st Qu.:0.0000   1st Qu.: 73.0  
##  Median :137.0   Median :1.0000   Median :0.0000   Median :115.0  
##  Mean   :136.6   Mean   :0.6488   Mean   :0.3211   Mean   :130.3  
##  3rd Qu.:140.0   3rd Qu.:1.0000   3rd Qu.:1.0000   3rd Qu.:203.0  
##  Max.   :148.0   Max.   :1.0000   Max.   :1.0000   Max.   :285.0  
##   DEATH_EVENT    
##  Min.   :0.0000  
##  1st Qu.:0.0000  
##  Median :0.0000  
##  Mean   :0.3211  
##  3rd Qu.:1.0000  
##  Max.   :1.0000
\end{verbatim}

\end{document}
